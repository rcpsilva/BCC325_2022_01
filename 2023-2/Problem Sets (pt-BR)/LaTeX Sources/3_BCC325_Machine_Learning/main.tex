\documentclass{article}
\usepackage[utf8]{inputenc}
\usepackage[margin=1.2in]{geometry}
\usepackage{hyperref}

\usepackage{tikz}
\usetikzlibrary{positioning}

\usepackage{natbib}
\usepackage{graphicx}
\usepackage{amsmath}
\usepackage{listings}
\usepackage{xcolor}


\definecolor{codegreen}{rgb}{0,0.6,0}
\definecolor{codegray}{rgb}{0.5,0.5,0.5}
\definecolor{codepurple}{rgb}{0.58,0,0.82}
\definecolor{backcolour}{rgb}{0.95,0.95,0.92}
\definecolor{deepblue}{rgb}{0,0,0.5}
\definecolor{deepred}{rgb}{0.6,0,0}
\definecolor{deepgreen}{rgb}{0,0.5,0}

\lstdefinestyle{mystyle}{
    backgroundcolor=\color{white},   
    commentstyle=\color{codegreen},
    keywordstyle=\color{deepblue},
    numberstyle=\tiny\color{codegray},
    stringstyle=\color{deepgreen},
    emph={Agent,__init__,act,self,union,exists, scope},
    emphstyle=\color{deepred},
    basicstyle=\ttfamily\footnotesize,
    breakatwhitespace=false,         
    breaklines=true,                 
    captionpos=b,                    
    keepspaces=true,                 
    numbers=left,                    
    numbersep=5pt,                  
    showspaces=false,                
    showstringspaces=false,
    showtabs=false,                  
    tabsize=3
}

\lstset{style=mystyle}

\title{\vspace{-2 cm}Universidade Federal de Ouro Preto \\ BCC 325 - Inteligência Artificial \\ Aprendizado de Máquina}
\author{Prof. Rodrigo Silva}
\date{}


\begin{document}

\maketitle

\section{Leitura}

\begin{itemize}
    \item Capítulo 7 do Livro\textit{ Artificial Intelligence: Foundations of Computational Agents,  3rd Edition} disponível em \textit{https://artint.info/}
    \item Aquele que tudo sabe, tudo vê e nada teme.
\end{itemize}

\section{Questões}

\begin{enumerate}
    \item No contexto da disciplina de aprendizado de máquina, o que significa ``aprender''?
    
    \item Quais componentes fazem parte de qualquer problema de aprendizado de máquina? (Ver. Seção 7.1 de \url{https://artint.info/})
    
    \item Para cada tipo de aprendizado abaixo, (i) Apresente uma descrição, (ii) Explique o funcionamento (Quais são as entreadas? Quais são as saídas? Como as saídas são obtidas da entrada) (iii) Apresente DUAS aplicações. (Obs: Não descrever nenhum algoritmo específico. Apresentar cada tipo de aprendizado como um estrutura básica que os algoritmos específicos devem seguir.)
    
    \begin{enumerate}
        \item Aprendizado não-supervisionado.
        \item Aprendizado supervisionado.
        \item Aprendizado por reforço.
        \item Aprendizado online.
        \item Aprendizado offline.
    \end{enumerate}
\end{enumerate}

\end{document}

