\documentclass{article}
\usepackage[utf8]{inputenc}
\usepackage[margin=1.2in]{geometry}
\usepackage{hyperref}

\PassOptionsToPackage{usenames,dvipsnames,svgnames}{xcolor}  
\usepackage{tikz}
\usetikzlibrary{arrows,positioning,automata}

\usepackage{natbib}
\usepackage{graphicx}
\usepackage{amsmath}
\usepackage{listings}
\usepackage{xcolor}

\usepackage{tikz}
\usetikzlibrary{arrows.meta, positioning, shapes.geometric}

\definecolor{codegreen}{rgb}{0,0.6,0}
\definecolor{codegray}{rgb}{0.5,0.5,0.5}
\definecolor{codepurple}{rgb}{0.58,0,0.82}
\definecolor{backcolour}{rgb}{0.95,0.95,0.92}
\definecolor{deepblue}{rgb}{0,0,0.5}
\definecolor{deepred}{rgb}{0.6,0,0}
\definecolor{deepgreen}{rgb}{0,0.5,0}

\lstdefinestyle{mystyle}{
    backgroundcolor=\color{white},   
    commentstyle=\color{codegreen},
    keywordstyle=\color{deepblue},
    numberstyle=\tiny\color{codegray},
    stringstyle=\color{deepgreen},
    emph={Agent,__init__,act,self,union,exists, scope},
    emphstyle=\color{deepred},
    basicstyle=\ttfamily\footnotesize,
    breakatwhitespace=false,         
    breaklines=true,                 
    captionpos=b,                    
    keepspaces=true,                 
    numbers=left,                    
    numbersep=5pt,                  
    showspaces=false,                
    showstringspaces=false,
    showtabs=false,                  
    tabsize=3
}

\lstset{style=mystyle}

\title{\vspace{-2 cm} Universidade Federal de Ouro Preto \\ BCC 325 - Inteligência Artificial \\ Lista 5 \\ Prof. Rodrigo Silva}
\date{}


\begin{document}

\maketitle

\section{Leitura Recomendada}

\begin{itemize}
    \item Capítulos 5,7,8 \url{https://artint.info/3e/html/ArtInt3e.html}
\end{itemize}

\section{Questões}

\begin{enumerate}

    \item Que tipo de problema resolvemos com o algoritmo de busca em largura? Qual a complexidade de tempo e espaço deste algoritmo?
    
    \item Que tipo de problema resolvemos com o algoritmo $A^*$? Qual a complexidade de tempo e espaço deste algoritmo?
    
    \item Quais são os componentes de um problema de satisfação de restrições?
    
    \item Considere a seguinte base de conhecimento (KB):
    \vspace{-0.5 cm}
    
    \begin{center}
        \begin{align*}
         a & \leftarrow b \wedge c. \\ 
         b & \leftarrow e. \\ 
         b & \leftarrow d. \\ 
         c &. \\ 
         d & \leftarrow h. \\ 
         e &. \\
         g & \leftarrow a \wedge b  \wedge e. \\
         f & \leftarrow a \wedge b. \\  
        \end{align*}
    \end{center}
    
    
    \begin{enumerate} 
        \item Apresente um modelo desta base de conhecimento.
        \item Apresente uma interpretação que não é um modelo desta base de conhecimento.
        \item Mostre um prova bottom-up para esta base de conhecimento. 
        \item Apresente uma prova top-down para a pergunta $ask$ $f$.
    \end{enumerate}
    
    \item Para cada um dos algoritmos de aprendizado de máquina abaixo, responda:

    \begin{enumerate}
        \item Quais são os principais componentes?
        \item Como os parâmetros do modelos são aprendidos?
    \end{enumerate}

    \begin{itemize}
        \item Regressão Linear
        \item Regressão Logística
        \item Árvores de Decisão
        \item Redes Neurais Artificiais
    \end{itemize}



    
    \end{enumerate}

\end{document}

